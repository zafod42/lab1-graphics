\documentclass[14pt, final]{extarticle}
\usepackage[utf8]{inputenc}
\usepackage[russianb]{babel}
\usepackage{vmargin}
\usepackage{graphicx}
\setpapersize{A4}
\setmarginsrb{3cm}{1cm}{2cm}{2cm}{0pt}{0mm}{0pt}{13mm}
\usepackage{indentfirst}
\usepackage{listings}
\usepackage{xcolor}
\definecolor{bluekeywords}{rgb}{0,0,1}
\definecolor{greencomments}{rgb}{0,0.5,0}
\definecolor{redstrings}{rgb}{0.64,0.08,0.08}
\lstset{
    language=C++,
    columns=fullflexible,
    frame=single,
    breaklines=true,
    numbers=left,
    showstringspaces=false,
    postbreak=\mbox{\textcolor{red}{$\hookrightarrow$}\space},
    commentstyle=\color{greencomments},
    breakatwhitespace=true,
    keywordstyle=\color{bluekeywords},
    stringstyle=\color{redstrings},
}
\sloppy
\title{Лабораторная работа 1. Вариант 1}
\author{Калабин Павел Павлович 5130904/20103}
\date{\today}
\begin{document}
\maketitle
\section{Постановка задачи}

Целью работы является ознакомление с основами векторной графики и получение 
навыков работы с базовыми функциями графического API и трехмерными графическими примитивами.
Требуется при помощи стандартных функций бибилиотеки (OpenGL/Vulkan или DirectX) изобразить 
указанные объекты и произвести необходимые преобразования.

\begin{enumerate}
    \item Изобразить каркасный конус и каркасную сферу, расположенные на некотором расстоянии друг от друга.
    \item Совместить центр основания конуса и центр сферы.
    \item Изобразить тор и цилиндр. Размеры и местоположение примитивов задать самостоятельно.
    \item Выполнить последовательно сначала поворот цилиндра вокруг оси Х, а затем растяжение тора в 2 раза.
\end{enumerate}

\section {Ход работы}

В качестве среды выполнения работы была выбрана библиотека OpenGL.
Для выполнения работы были использованы примитивы из библиотеки OpenGL
Utility (GLU) и OpenGL Utility Toolkit (GLUT).

\subsection{Настройка OpenGL}

Для корректной работы и отрисовки примитивов необходимо настроить OpenGL.
Для этого при помощи стека матриц были созданы 
\textit{матрица проекции (projection matrix)} и \textit{видовая матрица (view matrix)}.
Кратко работу со стеком матриц можно описать следующим образом: 

\begin{enumerate}
    \item Загружается матрица, с которой предполагается производить операции,
    \item При помощи функций библиотеки эта матрица умножается справа на изменяющую матрицу,
    \item На стеке остаётся преобразованная матрица с необходимыми нам характеристиками. 
\end{enumerate}

Подробнее работа со стеком матриц будет рассмотрена на примере построения сцены.

\textit{Матрица проекции} отвечает за проекцию трёхмерного пространства на двумерное
пространство экрана и за отсечение тех объектов, которые не находятся в поле зрения.

Существует несколько видов матриц проекции, например \textit{ортографическая проекция},
такая проекция переносит объекты <<как есть>> без учета перспективы.
Второй вид проекции это \textit{проекция с перспективой}, она позволяет отобразить
объекты с учетом их положения в пространстве так, как они бы выглядели при
взгляде на них с позиции камер.

В работе была использована матрица проекции с перспективой из библиотеки GLU,
которая имеет дополнительный параметр угла обзора (FOV).

\textit{Видовая матрица} отвечает за преобразование мировых координат в пространство
координат камеры, эта матрица как бы перемещает точку наблюдения в центр
камеры.

В качестве такой матрицы была использована матрица LookAt, которую
предоставляет функция \texttt{gluLookAt}.

\subsection{Сцены}

Была составлена сцена из синей сферы и фиолетового конуса (см. рис.\,\ref{scene11_figure}),
для этого использованы функции \texttt{glutWireCone} и \texttt{glutWireSphere}.
Эти функции используют внутреннюю реализацию из библиотеки GLU, в частности
реализация конуса представляет собой вызов функции отрисовки циллиндра с
нулевым параметром радиуса верхней части, так что верхний радиус циллиндра
вырождается в точку.

Для создания сцены необходимо использовать стек матриц, для представления
каждого примитива в виде набора вершин и манипуляции этими объектами.
При создании сцены на стек матриц заносится матрица, отвечающая за
трансформации сцены в целом (например, вращение целой сцены). Далее по
очередно заносятся матрицы, отвечающие за трансформации над объектами и
описываются соотвествующие им примитивы.

\subsection{Анимации}

Для анимирования сцен введены параметры, например смещение
сферы на некоторую позицию. Параметры сцен изменяются с течением времени 
в функциях \texttt{animateX}, которые определяют анимации для соответсвующей сцены.

\subsection{Демонстрация работы программы}

Далее на рисунках с \,\ref{scene11_figure} по \,\ref{scene14_figure} показана работа
программы и вид сцены №1.
На рисунках с \,\ref{scene21_figure} по \,\ref{scene24_figure} демонстрируется сцена №2.

\begin{figure}[h]
    \includegraphics[width=0.9\textwidth]{images/scene1_1}
    \caption{Сцена 1. Начальное состояние}
    \label{scene11_figure}
\end{figure}
\begin{figure}[h!]
    \includegraphics[width=0.9\textwidth]{images/scene1_2}
    \caption{Сцена 1. Перенос сферы}
    \label{scene12_figure}
\end{figure}
\begin{figure}[h!]
    \includegraphics[width=0.9\textwidth]{images/scene1_3}
    \caption{Сцена 1. Перенос сферы}
    \label{scene13_figure}
\end{figure}
\begin{figure}[h!]
    \includegraphics[width=0.9\textwidth]{images/scene1_4}
    \caption{Сцена 1. Перенос сферы}
    \label{scene14_figure}
\end{figure}

\begin{figure}[h!]
    \includegraphics[width=0.9\textwidth]{images/scene2_1}
    \caption{Сцена 2. Начальное состояние}
    \label{scene21_figure}
\end{figure}
\begin{figure}[h!]
    \includegraphics[width=0.9\textwidth]{images/scene2_2}
    \caption{Сцена 2. Вращение циллиндра}
    \label{scene22_figure}
\end{figure}
\begin{figure}[h!]
    \includegraphics[width=0.9\textwidth]{images/scene2_3}
    \caption{Сцена 2. Начало увеличения тора в 2 раза}
    \label{scene23_figure}
\end{figure}
\begin{figure}[h!]
    \includegraphics[width=0.9\textwidth]{images/scene2_4}
    \caption{Сцена 2. Тор увеличен в 2 раза}
    \label{scene24_figure}
\end{figure}

\newpage
\section*{Приложение}

\lstinputlisting[caption=Исходный текст программы, label=lst:main]{main.cc}
\end{document}
